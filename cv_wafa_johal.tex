

% Don't like 10pt? Try 11pt or 12pt
\documentclass[10pt]{article}
\RequirePackage[T1]{fontenc}
%\usepackage{pbox}
\usepackage[french]{babel}
\usepackage[utf8]{inputenc}
\usepackage{lmodern}
%\usepackage{times}
\usepackage[bitstream-charter,euro]{mathdesign}
\usepackage{calc}


\usepackage[shortcuts]{extdash}

% Layout: Puts the section titles on left side of page
\reversemarginpar



%% Use these lines for letter-sized paper
\usepackage[paper=a4paper,
            %includefoot, % Uncomment to put page number above margin
            marginparwidth=1in,     % Length of section titles
            marginparsep=.05in,       % Space between titles and text
            margin=1cm,               % 1 inch margins
            includemp]{geometry}

%% Use these lines for A4-sized paper
%\usepackage[paper=a4paper,
%            %includefoot, % Uncomment to put page number above margin
%            marginparwidth=30.5mm,    % Length of section titles
%            marginparsep=1.5mm,       % Space between titles and text
%            margin=25mm,              % 25mm margins
%            includemp]{geometry}

%% More layout: Get rid of indenting throughout entire document
\setlength{\parindent}{0in}

% Provides special list environments and macros to create new ones
\usepackage[shortlabels]{enumitem}

% Simpler bibsections for CV sections
% (thanks to natbib for inspiration)
%
% * For lists of references with hanging indents and no numbers:
%
%   \begin{bibsection}
%       \item ...
%   \end{bibsection}
%
% * For numbered lists of references (with hanging indents):
%
%   \begin{bibenum}
%       \item ...
%   \end{bibenum}
%
%   Note that bibenum numbers continuously throughout. To reset the
%   counter, use
%
%   \restartlist{bibenum}
%
%   at the place where you want the numbering to reset.

\makeatletter
\newlength{\bibhang}
\setlength{\bibhang}{1em}
\newlength{\bibsep}
 {\@listi \global\bibsep\itemsep \global\advance\bibsep by\parsep}
\newlist{bibsection}{itemize}{3}
\setlist[bibsection]{label=,leftmargin=\bibhang,%
        itemindent=-\bibhang,
        itemsep=\bibsep,parsep=\z@,partopsep=0pt,
        topsep=0pt}
\newlist{bibenum}{enumerate}{3}
\setlist[bibenum]{label=[\arabic*],resume,leftmargin={\bibhang+\widthof{[999]}},%
        itemindent=-\bibhang,
        itemsep=\bibsep,parsep=\z@,partopsep=0pt,
        topsep=0pt}
\let\oldendbibenum\endbibenum
\def\endbibenum{\oldendbibenum\vspace{-.6\baselineskip}}
\let\oldendbibsection\endbibsection
\def\endbibsection{\oldendbibsection\vspace{-.6\baselineskip}}
\makeatother

%% Reference the last page in the page number
%
% NOTE: comment the +LP line and uncomment the -LP line to have page
%       numbers without the ``of ##'' last page reference)
%
% NOTE: uncomment the \pagestyle{empty} line to get rid of all page
%       numbers (make sure includefoot is commented out above)
%
\usepackage{fancyhdr,lastpage}
\pagestyle{fancy}
\pagestyle{empty}      % Uncomment this to get rid of page numbers
\fancyhf{}\renewcommand{\headrulewidth}{0pt}
\fancyfootoffset{\marginparsep+\marginparwidth}
\newlength{\footpageshift}
\setlength{\footpageshift}
          {0.5\textwidth+0.5\marginparsep+0.5\marginparwidth-2in}
\lfoot{\hspace{\footpageshift}%
       \parbox{4in}{\, \hfill %
                    \arabic{page} of \protect\pageref*{LastPage} % +LP
%                    \arabic{page}                               % -LP
                    \hfill \,}}

% Finally, give us PDF bookmarks
\usepackage{color,hyperref}
\definecolor{darkblue}{rgb}{0.0,0.0,0.3}
\hypersetup{colorlinks,breaklinks,
            linkcolor=darkblue,urlcolor=darkblue,
            anchorcolor=darkblue,citecolor=darkblue}

%%%%%%%%%%%%%%%%%%%%%%%% End Document Setup %%%%%%%%%%%%%%%%%%%%%%%%%%%%


%%%%%%%%%%%%%%%%%%%%%%%%%%% Helper Commands %%%%%%%%%%%%%%%%%%%%%%%%%%%%

%%% HEADING AT TOP OF CURRICULUM VITAE

% The title (name) with a horizontal rule under it
% (optional argument typesets an object right-justified across from name
%  as well)
%
% Usage: \makeheading{name}
%        OR
%        \makeheading[right_object]{name}
%
% Place at top of document. It should be the first thing.
% If ``right_object'' is provided in the square-braced optional
% argument, it will be right justified on the same line as ``name'' at
% the top of the CV. For example:
%
%      \makeheading[\emph{Curriculum vitae}]{Your Name}
%
% will put an emphasized ``Curriculum vitae'' at the top of the document
% as a title. Likewise, a picture could be included:
%
%   \makeheading[{\includegraphics[height=1.5in]{my_picture}}]{Your Name}
%
% the picture will be flush right across from the name. For this example
% to work, make sure the extra set of curly braces is included. Also
% makes ure that \usepackage{graphicx} is somewhere in the preamble.
\newcommand{\makeheading}[2][]%
        {\hspace*{-\marginparsep minus \marginparwidth}%
         \begin{minipage}[t]{\textwidth+\marginparwidth+\marginparsep}%
             {\large \bfseries #2 \hfill #1}\\[-0.15\baselineskip]%
                 \rule{\columnwidth}{1pt}%
         \end{minipage}}

%%% SECTION HEADINGS

% The section headings. Flush left in small caps down pseudo-margin.
%
% Usage: \section{section name}
\renewcommand{\section}[1]{\pagebreak[3]%
    \vspace{1.3\baselineskip}%
    \phantomsection\addcontentsline{toc}{section}{#1}%
    \noindent\llap{\scshape\smash{\parbox[t]{\marginparwidth}{\hyphenpenalty=10000\raggedright #1}}}%
    \vspace{-\baselineskip}\par}

%%% LISTS

% This macro alters a list by removing some of the space that follows the list
% (is used by lists below)
\newcommand*\fixendlist[1]{%
    \expandafter\let\csname preFixEndListend#1\expandafter\endcsname\csname end#1\endcsname
    \expandafter\def\csname end#1\endcsname{\csname preFixEndListend#1\endcsname\vspace{-0.6\baselineskip}}}

% These macros help ensure that items in outer-type lists do not get
% separated from the next line by a page break
% (they are used by lists below)
\let\originalItem\item
\newcommand*\fixouterlist[1]{%
    \expandafter\let\csname preFixOuterList#1\expandafter\endcsname\csname #1\endcsname
    \expandafter\def\csname #1\endcsname{\let\oldItem\item\def\item{\pagebreak[2]\oldItem}\csname preFixOuterList#1\endcsname}
    \expandafter\let\csname preFixOuterListend#1\expandafter\endcsname\csname end#1\endcsname
    \expandafter\def\csname end#1\endcsname{\let\item\oldItem\csname preFixOuterListend#1\endcsname}}
\newcommand*\fixinnerlist[1]{%
    \expandafter\let\csname preFixInnerList#1\expandafter\endcsname\csname #1\endcsname
    \expandafter\def\csname #1\endcsname{\let\oldItem\item\let\item\originalItem\csname preFixInnerList#1\endcsname}
    \expandafter\let\csname preFixInnerListend#1\expandafter\endcsname\csname end#1\endcsname
    \expandafter\def\csname end#1\endcsname{\csname preFixInnerListend#1\endcsname\let\item\oldItem}}

% An itemize-style list with lots of space between items
%
% Usage:
%   \begin{outerlist}
%       \item ...    % (or \item[] for no bullet)
%   \end{outerlist}
\newlist{outerlist}{itemize}{3}
    \setlist[outerlist]{label=\enskip\textbullet,leftmargin=*}
    \fixendlist{outerlist}
    \fixouterlist{outerlist}

% An environment IDENTICAL to outerlist that has better pre-list spacing
% when used as the first thing in a \section
%
% Usage:
%   \begin{lonelist}
%       \item ...    % (or \item[] for no bullet)
%   \end{lonelist}
\newlist{lonelist}{itemize}{3}
    \setlist[lonelist]{label=\enskip\textbullet,leftmargin=*,partopsep=0pt,topsep=0pt}
    \fixendlist{lonelist}
    \fixouterlist{lonelist}

% An itemize-style list with little space between items
%
% Usage:
%   \begin{innerlist}
%       \item ...    % (or \item[] for no bullet)
%   \end{innerlist}
\newlist{innerlist}{itemize}{3}
    \setlist[innerlist]{label=\enskip\textbullet,leftmargin=*,parsep=0pt,itemsep=0pt,topsep=0pt,partopsep=0pt}
    \fixinnerlist{innerlist}

% An environment IDENTICAL to innerlist that has better pre-list spacing
% when used as the first thing in a \section
%
% Usage:
%   \begin{loneinnerlist}
%       \item ...    % (or \item[] for no bullet)
%   \end{loneinnerlist}
\newlist{loneinnerlist}{itemize}{3}
    \setlist[loneinnerlist]{label=\enskip\textbullet,leftmargin=*,parsep=0pt,itemsep=0pt,topsep=0pt,partopsep=0pt}
    \fixendlist{loneinnerlist}
    \fixinnerlist{loneinnerlist}

%%% EXTRA SPACE

% To add some paragraph space between lines.
% This also tells LaTeX to preferably break a page on one of these gaps
% if there is a needed pagebreak nearby.
\newcommand{\blankline}{\quad\pagebreak[3]}
\newcommand{\halfblankline}{\quad\vspace{-0.5\baselineskip}\pagebreak[3]}

%%% FORMATTING MACROS

% Provides a linked \doi{#1} that links doi:#1 to http://dx.doi.org/#1
\usepackage{doi}
% To change the text before the DOI, adjust this command
%\renewcommand\doitext{doi:}

% Provides a linked \url{#1} that doesn't require escape characters
\usepackage{url}

% You can adjust the style \url{} uses here:
% (options are: same, rm, sf, tt; defaults to tt)
\urlstyle{same}

% For \email{ADDRESS}, links ADDRESS to the url mailto:ADDRESS
% (uncomment to typeset the e\-/mail address in typewriter font;
%  otherwise, will be typeset in the \urlstyle above)
%\DeclareUrlCommand\emaillink{\urlstyle{tt}}
\providecommand*\emaillink[1]{\nolinkurl{#1}}
\providecommand*\email[1]{\href{mailto:#1}{\emaillink{#1}}}

\providecommand\BibTeX{{B\kern-.05em{\sc i\kern-.025em b}\kern-.08em \TeX}}
\providecommand\Matlab{\textsc{Matlab}}

% Custom hyphenation rules for words that LaTeX has trouble with
\hyphenation{bio-mim-ic-ry bio-in-spi-ra-tion re-us-a-ble pro-vid-er Media-Wiki}

%%%%%%%%%%%%%%%%%%%%%%%% End Helper Commands %%%%%%%%%%%%%%%%%%%%%%%%%%%

%%%%%%%%%%%%%%%%%%%%%%%%% Begin CV Document %%%%%%%%%%%%%%%%%%%%%%%%%%%%

\begin{document}
\makeheading{Wafa Johal }

\section{Contact Information}

% NOTE: Mind where the & separators and \\ breaks are in the following
%       table. Table is one row made up of three parboxes. The left
%       parbox has address info, the middle parbox has a vertical bar,
%       and the right parbox has phone and electronic contact
%       information.
%
% MACROS: \rcollength is the width of the right column of the table
%             (adjust it to your liking; default is 1.85in).
%         \spacewidth is width of area between left and right boxes.
%
\newlength{\rcollength}\setlength{\rcollength}{1.85in}%
\newlength{\spacewidth}\setlength{\spacewidth}{20pt}
%
\begin{tabular}[t]{@{}p{\textwidth-\rcollength-\spacewidth}@{}p{\spacewidth}@{}p{\rcollength}}%

% Address box
\parbox{\textwidth-\rcollength-\spacewidth}{%
PostDoctoral Scientist\\
\href{http://chili.epfl.ch}{CHILI Lab} - \href{http://epfl.ch}{Ecole Polytechnique Federal de Lausanne}\\
\\
EPFL IC ISIM CHILI \\
RLC D1 740 (Rolex Learning Center) \\
Station 20 \\
CH-1015 Lausanne, SUISSE}

&
% Uncomment to add a vertical bar in middle of contact information
%{\vrule width 0.5pt}
\parbox[m][5\baselineskip]{\spacewidth}{} &

% Non-snail-mail contact information
\parbox{\rcollength}{%
\textit{Mobile:} +33-6 69 69 48 16 \\
\textit{E-mail:} \email{wafa.johal@gmail.com}\\
\textit{WWW:} \href{http://wafa.johal.fr/}{wafa.johal.fr}}

\end{tabular}

%%
%% In modern CV's, it seems like ``Objective'' is frowned upon. Instead,
%% incorporate it into a well-constructed cover letter. The ``More
%% information'' can go at the end of the CV, but it should not distract
%% from the section giving references available to contact.
%%
%
% \section{Objective}
%
% Placement in an academic position (i.e., faculty, postdoctoral, or
% research scientist) that allows for advanced research in distributed
% complex adaptive systems (i.e., modeling, analysis, design, and
% verification) with a particular focus on the control of engineered
% agents (e.g., for communications, control, software, electronics, and
% sustainability) and the analysis of biological phenomena (e.g.,
% self-organization, ecological rationality)
% \begin{innerlist}
% \item More information and auxiliary documents can be found at\\\url{http://www.tedpavlic.com/facjobsearch/}
% \end{innerlist}

\section{Research Interests}
I am interested in personalization using  non-verbal cues of communication during Human-Robot Interaction. 
I believe that personalization of robots' behaviours will improve acceptability. I work with robots having various embodiment and try to base my models on social sciences and cognitive sciences. 
I also have interest in affective computing, learning, symbolic reasoning and integration of companion robots in smart homes. 
I have conducted child-robot interactions experiments from protocole design to data analysis.

\section{Academia}

\href{http://www.univ-grenoble-alpes.fr/}{\textbf{University of Grenoble Alps}}, \href{http://liglab.fr}{Laboratory of Informatics of Grenoble} - \href{http://magma.imag.fr}{MAGMA Team} France
\begin{outerlist}

\item[] Ph.D. Student in Computer Sciences, October 2012 - October 2015
        \begin{innerlist}
        \item Thesis Topic: \emph{Behavioural Styles for Human-Robot Interaction: Towards Plasticity in HRI}
        \item Advisers:
              \href{http://magma.imag.fr/content/sylvie-pesty}
                   {Prof. Sylvie Pesty} \& 
              \href{http://iihm.imag.fr/calvary/}
                   {Prof. Gaëlle Calvary}
        \item Area of Study : Affective Computing, Human-Robot Interaction, Personalisation, Plasticity
        \end{innerlist}

\end{outerlist}
\vspace{0.5cm}
\href{https://www.ujf-grenoble.fr/?language=en}{\textbf{University Joseph Fourier}, Grenoble}, France
\begin{outerlist}
\item[] M.S.,
        %\href{http://www.ece.osu.edu/}
             {Computer Sciences: Graphics, Vision \& Robotics}, June 2012
        \begin{innerlist}
        \item Thesis Topic: \emph{Multi-modal detection of intention of interaction by a companion robot}
        \item Adviser:
              \href{https://team.inria.fr/prima/vaufreydaz/}
                   {Dr. Dominique Vaufreydaz}
        \item Area of Study : Activity Recognition, Multi-modal Data Fusion
        \end{innerlist}

\end{outerlist}
\vspace{0.5cm}
\href{http://www.upmf-grenoble.fr/}{\textbf{ Pierre-Mendès France University}, Grenoble},France
\begin{outerlist}

\item[] B.S. B.A. Mathematics and Informatics applied to Cognitive Sciences, June 2010
        \begin{innerlist}
        \item \emph{Topped, Summa cum Laude}
        \item Mathematics : Linear Algebra, Logic, Probabilities, Statistics
        \item Computer Sciences : Algorithms \& Programming
        \item Cognitive Sciences: Memory, Learning \& Perception
        \end{innerlist}
        Exchange Student for one year in Washington College, MA (U.S.A.), 2009-2010

\end{outerlist}



% \section{Submitted Journal Publications}
%
% % Add a little space to nudge next ``Ref'd Journal Publications'' marginpar
% % down to make room for tall ``Submitted Journal Publications''
% % marginpar. If there are enough submitted journal publications, this
% % space will not be needed (and should be removed).
% \vspace{0.1in}

\section{ Journal Publications}

\begin{bibenum}
    \item Vaufreydaz D., \textbf{Johal W}., and Combe C.,
   Starting engagement detection towards a companion robot using multimodal features, \emph{Robotics and Autonomous} Systems, January 2015, \href{http://dx.doi.org/10.1016/j.robot.2015.01.004}{ISSN 0921-8890}.
\end{bibenum}


\vspace{0.1in}

\section{International Conference Publications}

\begin{bibenum}
	
\item Ta V.C, \textbf{Johal W}., Portaz M., Castelli E., Vaufreydaz D., The Grenoble system for the Social Touch Challenge at ICMI 2015. International Conference on Multimodal Interaction. (ICMI2015). [In accepted] 
	
	
\item  \textbf{Johal W}., Calvary G., Pesty S., Non-verbal Signals in HRI: Interfer- ence in Human Perception. International Conference on Social Robotics. (ICSR2015). [In accepted] 
	
	
\item  \textbf{Johal W.}, Pellier D., Adam C., Fiorino H., Pesty S., A Cognitive and Affective Architecture for Social Human-Robot Interaction. HRI'15 Extended Abstracts. \href{http://doi.acm.org/10.1145/2701973.2702006}{DOI=10.1145/2701973.2702006 }  [Award Nominee]	

\item  \textbf{Johal W.}, Robots Interacting with Style. HRI'15 Extended Abstracts.\href{http://doi.acm.org/10.1145/2701973.2702706}{DOI=10.1145/2701973.2702706}   

\item \textbf{Johal W.}, Calvary G., Pesty S., Toward Companion Robots Behaving with Style. \emph{In Proceedings of the 2014 IEEE International Symposium on Robot and Human Interactive Communication} (ROMAN). Edinburgh. 2014. \href{ http://ieeexplore.ieee.org/stamp/stamp.jsp?tp=&arnumber=6926393&isnumber=6926219}{doi: 10.1109/ROMAN.2014.6926393}

\item \textbf{Johal W.}, Adam C., Fiorino H., Pesty, S., Duhaut D.. Acceptability of a companion
robot for children in daily life
situations \emph{5th IEEE Conference on Cognitive InfoCommunications}. (CogInfoCom) Nov 2014. Vietri, Italy. \href{http://ieeexplore.ieee.org/stamp/stamp.jsp?tp=&arnumber=7020474&isnumber=7020399}{doi: 10.1109/CogInfoCom.2014.7020474}


\item \textbf{Johal W.}, Calvary G. and  Pesty S.. A Robot with Style because you are Worth it!.  \emph{In CHI '14 Extended Abstracts on Human Factors in Computing Systems} (CHI EA '14). Toronto, Canada. \href{http://doi.acm.org/10.1145/2559206.2581229}{DOI=10.1145/2559206.2581229}

\item \textbf{Johal W.}, Dugdale J.,  Pesty S.. Modelling interactions in a mixed agent world. In Proceeding of 25th \emph{European Modeling and Simulation Symposium} (EMSS). Athens, Greece, 23-25 Sept 2013

\item  \textbf{Benkaouar (Johal) W.}, Vaufreydaz D.. Multi-sensors engagement detection with a robot companion in a home environment. \emph{Workshop on Assistance and Service Robotics in a Human Environment at IEEE International Conference on Intelligent Robots and Systems} (IROS2012), Vilamoura, Algarve - Portugal, oct 2012.

  

\end{bibenum}

\section{French Conference Publications}

\begin{bibenum}
\item \textbf{Johal W.}, Pesty, S., Calvary G., Des Styles pour une Personnalisation de l’Interaction Homme-Robot. \emph{Journées Nationales de la Robotique Interactive} (JNRI). Nov. 2014. Toulouse, France

\item \textbf{Johal W.}, Pesty S. Calvary G., Les Styles pour la Plasticité des Robots Compagnons. \emph{Workshop Affect, Compagnon Artificiel, Interaction} (WACAI) 2014. Rouen, France 

\item Adam C.,  \textbf{Johal W.}, Ben-Farhat I., Jost C., Fiorono H.,  Pesty S. , Duhaut D.. Acceptabilité d'un robot compagnon dans des situations de la vie quotidienne. \emph{ Deuxième conférence III - Intercompréhension de l'intraspécifique à l'interspécifique}. Lorient, 30 Sept - 1Oct 2013

  

\end{bibenum}

\section{Other \\ Publications}

\begin{bibenum}
\item \textbf{Johal W.}, Pesty S.,  Calvary G.. Expressing Parenting Styles with Companion Robots. \emph{Workshop on Applications for Emotional Robots at Human-Robot Interaction conference} (HRI’14) . Bielefeld. 2014

\end{bibenum}

%
%\section{Papers in Preparation}
%
%\begin{bibenum}
%   
%	\item  Johal W., Pellier D., Adam C., Fiorino H., Pesty S.. A Cognitive and Affective Architecture for Social Human-Robot Interaction. \emph{Late Breaking Report Human-Robot Interaction conference} (HRI'2015)	[accepted]
%	\item  Johal W.. Robots Interacting with Style. \emph{Pioneers of Human-Robot Interaction Conference} (HRI Pioneers '2015)	[accepted]
%	
%	
%\end{bibenum}

\section{Intern Tutoring}

\begin{itemize}
    \item[] \textbf{Jérémy Lascaux},
        Under-Graduate Student in Computer Sciences, IUT2- UPMF.
        Developing an Affective Math Game with Nao Robot.
        Spring 2014.
	  \item[] \textbf{Anshul Singh Parihar},
	  Electrical Engineering, Indian Institute of Technology, Jodhpur India
	  Agent-Based Simulation of Human-Robot Interaction for Experiment Design.
	  Spring 2014.

\end{itemize}

\section{Teaching}

\href{http://www.grenoble-inp.fr/grenoble-institute-of-technology-9224.kjsp?RH=INPG_FR}{\textbf{Grenoble Institute of Technology}}, France
\begin{outerlist}
\item[] \textit{Teaching Assistant}%
    \hfill \textbf{October~2012 - July~2014}
    \begin{innerlist}
        \item Instructor for ACVL: Software’s Analysis, Conception and Validation.
            %
            \begin{innerlist}
                \item Methods for Requirement Analysis
                \item Software Design Patterns
                \item UML Diagrams and Object Constraint Language 
            \end{innerlist}
          \item Instructor for Introduction to Programming
            \begin{innerlist}
                 \item OCaml and Python
                 \item Begining of complexity analysis
             \end{innerlist}
    \end{innerlist}
\end{outerlist}
\halfblankline

\href{http://www.upmf-grenoble.fr/}{\textbf{Pierre Mendès-France University}},
Grenoble, France  \hfill \textbf{October 2013- January~2014}
\begin{outerlist}

\item[] %\textit{Teaching Assistant}%
   
    \begin{innerlist}
        \item Instructor for INF f-1: Undergraduate Beginner Programming, Lectures \& Practical Sessions
       
    \end{innerlist}

\end{outerlist}

\halfblankline

\href{https://www.ujf-grenoble.fr/}{\textbf{Joseph Fourrier University}},
Grenoble, France    \hfill \textbf{October 2011- January~2012}
\begin{outerlist}

\item[] %\textit{Teaching Assistant}%
 
    \begin{innerlist}
        \item Instructor for Biology Undergraduates:  C2i, Burautic, Basic HTML,  Lectures \& Practical Sessions
       
    \end{innerlist}

\end{outerlist}
\section{Academic Services}

\textbf{Conferences \& Events }
\begin{innerlist}
     \item Member of Organisation Committee,
            ``Workshop Affect, Compagnon Artificiel, Interaction '',
            \href{http://wacai2012.imag.fr/}{WACAI 2012} 
            Grenoble, France - 15 et 16 November 2012.
      \item Member of Organisation Committee,
           \href{http://persycup.imag.fr/}{PersyCup Robotic Challenge} 
                  Grenoble, France - 21 Mai 2015.
\end{innerlist}

\halfblankline

\textbf{Reviews }
\begin{innerlist}
\item \emph{Journal on Interaction Studies}, 2016
	  \item \emph{International Conference on Social Robotics (ICSR)}, 2015
	  \item \emph{International Conference on Human Robot Interaction (HRI)}, 2015, 2016
    \item \emph{IEEE International Symposium on Robot and Human Interactive Communication (RO-MAN)}, 2014, 2015
  
    
\end{innerlist}

%\halfblankline




\section{Language}
%Language:
%
\begin{innerlist}
    \item French : Mother tongue
    \item English : Fluent - TOEFL, 1 year in USA, Masters taught in English
\end{innerlist}



%Robotics Platforms:
%%
%\begin{innerlist}
%    \item \href{http://reeti.fr/index.php/fr/}{Reeti} : PC-Bot with mounted motorized and expressive head By \href{http://www.robopec.com/}{Robopec}
%    \item \href{http://www.robosoft.com/robotic-solutions/healthcare/kompai/index.html}{Kompai} :  a robot to assist seniors and dependent persons at home By \href{http://www.robosoft.com/}{Robosoft}
%    \item \href{http://www.aldebaran.com/fr/qui-est-nao}{Nao} : a small humanoid robot for child-robot interaction By \href{http://www.aldebaran.com/}{Aldebaran}
%\end{innerlist}
%
%\halfblankline
%
%Instrumentation, Control, Data Acquisition, Test, and Measurement:
%\begin{innerlist}
%     \item Kinect 1 \& 2: skeletons, depth, video, audio, faces and other features extraction
%     \item French Translation of \href{http://www.bartneck.de/2008/03/11/the-godspeed-questionnaire-series/}{Godspeed Questionnaire} and other questionnaires for HRI
%\end{innerlist}
%
%\halfblankline
%
%Computer Programming:
%%
%\begin{innerlist}
%    \item Python, C$+$$+$, Java, Bash scripting , 
%    Cmake, OCaml, Urbi, Prolog
%    \item OpenCV, Openhab OpenGL, ROS, Kinect SDK, Naoqi
%    \item Numerical Analysis :  R, Gnuplot, Numpy 
%    \item Version Control : Svn, Git
%\end{innerlist}
%
%\halfblankline
%
%Desktop Editing and Productivity Software:
%%
%\begin{innerlist}
%    \item  Eclipse, Geany, MS Visual Studio, OpenHab Designer, Qt Creator 
%    \item \LaTeX{},Office packs
%    \item GIMP, Inkscape, Audacity, iMovie
%\end{innerlist}
%
%Operating Systems:
%%
%\begin{innerlist}
%    \item  Linux (Ubuntu, Archlinux), Apple OS X, Microsoft Windows family,
%\end{innerlist}






%\section{Popular Press}
%
%\begin{bibsection}
%
%    \item Sigfried, Tom. ``If the world is a computer, life is an
%        algorithm'', \emph{Science News: Context}, June 18, 2014.
%        \url{https://www.sciencenews.org/blog/context/if-world-computer-life-algorithm}
%
%    \item ``The Free \& Unfree: Open Source Everywhere~-- How a Global
%        Coding Coalition Built an Open Source Superserver'',
%        \emph{Wired}, 12(06), June 2004.
%
%\end{bibsection}





\section{References Available to Contact}
\textbf{Prof.~  \href{http://magma.imag.fr/content/sylvie-pesty}{Sylvie  Pesty}}
(e\-/mail:~\href{mailto:Sylvie.Pesty@imag.fr}{Sylvie.Pesty@imag.fr})
%
\begin{innerlist}
    \item Professor at \href{http://www.iut2.upmf-grenoble.fr/-/}{University Institute of Technology} (IUT2-UPMF),
    \item[$\diamond$] Member of \href{http://magma.imag.fr/}{MAGMA Team}, LIG-Lab
    \item[$\star$] \emph{Prof. Pesty is my supervisor for my PhD.}
\end{innerlist}
\halfblankline

\textbf{Prof.~ \href{http://iihm.imag.fr/calvary/}{Gaëlle Calvary}}
(e\-/mail:~\href{mailto:Gaelle.Calvary@imag.fr}{Gaelle.Calvary@imag.fr})
%
\begin{innerlist}
    \item Professor at \href{http://ensimag.grenoble-inp.fr/}{ Ensimag software engineering school}  of the Grenoble Institute of Technology,
    \item[$\diamond$] Member of \href{http://iihm.imag.fr/calvary/}{Engineering of Computer Human Interaction Team}, LIG-Lab
    \item[$\star$] \emph{Prof. Calvary is my co-supervisor for my PhD.}
\end{innerlist}

\halfblankline

\textbf{Dr.~ \href{https://team.inria.fr/prima/vaufreydaz}{Dominique Vaufreydaz}}
(e\-/mail:~\href{mailto:Dominique.Vaufreydaz@inria.fr}{Dominique.Vaufreydaz@inria.fr})
%
\begin{innerlist}
    \item Associate Professor at \href{http://www.upmf-grenoble.fr/-/}{University Pierre Mendès-France} ,
    \item[$\diamond$] Member of \href{https://team.inria.fr/prima/}{PRIMA Team}, INRIA \& LIG-Lab
    \item[$\star$] \emph{Prof. Vaufreydaz was my supervisor during my Masters.}
\end{innerlist}

\section{}

\section{More Information}
More information and auxiliary documents can be found at\\%
\url{http://wafa.johal.fr/}.


\end{document}

%%%%%%%%%%%%%%%%%%%%%%%%%% End CV Document %%%%%%%%%%%%%%%%%%%%%%%%%%%%%

%----------------------------------------------------------------------%
% The following is copyright and licensing information for
% redistribution of this LaTeX source code; it also includes a liability
% statement. If this source code is not being redistributed to others,
% it may be omitted. It has no effect on the function of the above code.
%----------------------------------------------------------------------%
% Copyright (c) 2007, 2008, 2009, 2010, 2011 by Theodore P. Pavlic
%
% Unless otherwise expressly stated, this work is licensed under the
% Creative Commons Attribution-Noncommercial 3.0 United States License. To
% view a copy of this license, visit
% http://creativecommons.org/licenses/by-nc/3.0/us/ or send a letter to
% Creative Commons, 171 Second Street, Suite 300, San Francisco,
% California, 94105, USA.
%
% THE SOFTWARE IS PROVIDED "AS IS", WITHOUT WARRANTY OF ANY KIND, EXPRESS
% OR IMPLIED, INCLUDING BUT NOT LIMITED TO THE WARRANTIES OF
% MERCHANTABILITY, FITNESS FOR A PARTICULAR PURPOSE AND NONINFRINGEMENT.
% IN NO EVENT SHALL THE AUTHORS OR COPYRIGHT HOLDERS BE LIABLE FOR ANY
% CLAIM, DAMAGES OR OTHER LIABILITY, WHETHER IN AN ACTION OF CONTRACT,
% TORT OR OTHERWISE, ARISING FROM, OUT OF OR IN CONNECTION WITH THE
% SOFTWARE OR THE USE OR OTHER DEALINGS IN THE SOFTWARE.
%----------------------------------------------------------------------%
